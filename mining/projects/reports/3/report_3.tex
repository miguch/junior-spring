\documentclass[12pt]{article}
% Chinese Support
\usepackage{xeCJK}
\setCJKmainfont{STSong}
% \setmainfont{Times New Roman}

\usepackage{float}

% for symbol
\usepackage{gensymb}
% matrix
\usepackage{amsmath}

% For images
\usepackage{graphicx}
\graphicspath{ {./screenshoot/} }

\newcommand{\numpy}{{\tt numpy}}    % tt font for numpy

\topmargin -1.in
\textheight 9in
\oddsidemargin -.25in
\evensidemargin -.25in
\textwidth 7in


\usepackage{listings}
\usepackage{color}
\usepackage{xcolor}

\definecolor{dkgreen}{rgb}{0,0.6,0}
\definecolor{gray}{rgb}{0.5,0.5,0.5}
\definecolor{mauve}{rgb}{0.58,0,0.82}
\definecolor{textblue}{rgb}{.2,.2,.7}
\definecolor{textred}{rgb}{0.77,0,0}
\definecolor{textgreen}{rgb}{0,0.43,0}

% \setmonofont{FiraCode-Regular}
\lstset{frame=tb,
  language=C++,
  aboveskip=3mm,
  belowskip=3mm,
  showstringspaces=false,
  columns=flexible,
  basicstyle={\ttfamily},
  numbers=none,
  numberstyle=\tiny\color{gray},
  keywordstyle=\color{blue}\itshape,
  stringstyle=\color{mauve},
  breaklines=true,
  breakatwhitespace=true,
  commentstyle=\color{textred}\itshape,
  tabsize=3
}



% Content start below
\begin{document}

\author{陈铭涛\\16340024}
\title{数据挖掘周报 3}
% \date{\vspace{-5ex}}
\maketitle

\medskip

% ========== Begin answering questions here

\section{这两周所做工作}
\begin{enumerate}
    \item 下载并了解京东杯 用户对品类下店铺的购买预测 的数据集内容
    \item 查找包括去年 JDATA 比赛在内的购买预测方向的数据挖掘竞赛的内容,学习该类竞赛中的数据清洗、特征工程、训练集构建、模型验证以及模型选择的方法。
    \item 进行了初步的京东杯比赛的特征处理,目前初步的做法是将需要预测的时间段前90天内的用户操作、评论操作、时间分布与间隔,以及用户本身的信息作为特征,将特征集时间后的一周内的用户购买情况作为标签进行训练。目前特征处理的内容尚未完成,因此还未进行模型的训练。
\end{enumerate}

\section{遇到的问题}
\begin{enumerate}
    \item 京东杯比赛数据量较大,全部数据从 csv 读取后就需要 5G 以上内存,在 Jupyter Notebook 中偶尔某个 cell 运行出错后其使用的内存即使执行\lstinline{gc.collect()}也未能释放,在特征处理阶段中的内存占用经常到达 15G 以上,只能重启 kernel 来释放占用的内存,重新执行先前的操作,导致进度较慢。
\end{enumerate}


\section{下一阶段计划}

\begin{enumerate}
    \item 继续进行京东杯的内容。
    \item 期末考核的比赛出来后,与京东杯对比看看决定是否要继续进行京东杯的工作。
\end{enumerate}

\end{document}
\grid
\grid
