\documentclass[12pt]{article}
% Chinese Support
\usepackage{xeCJK}
\setCJKmainfont{STSong}
% \setmainfont{Times New Roman}

\usepackage{float}

% for symbol
\usepackage{gensymb}
% matrix
\usepackage{amsmath}

% For images
\usepackage{graphicx}
\graphicspath{ {./screenshoot/} }

\newcommand{\numpy}{{\tt numpy}}    % tt font for numpy

\topmargin -1.in
\textheight 9in
\oddsidemargin -.25in
\evensidemargin -.25in
\textwidth 7in


\usepackage{listings}
\usepackage{color}
\usepackage{xcolor}

\definecolor{dkgreen}{rgb}{0,0.6,0}
\definecolor{gray}{rgb}{0.5,0.5,0.5}
\definecolor{mauve}{rgb}{0.58,0,0.82}
\definecolor{textblue}{rgb}{.2,.2,.7}
\definecolor{textred}{rgb}{0.77,0,0}
\definecolor{textgreen}{rgb}{0,0.43,0}

% \setmonofont{FiraCode-Regular}
\lstset{frame=tb,
  language=C++,
  aboveskip=3mm,
  belowskip=3mm,
  showstringspaces=false,
  columns=flexible,
  basicstyle={\ttfamily},
  numbers=none,
  numberstyle=\tiny\color{gray},
  keywordstyle=\color{blue}\itshape,
  stringstyle=\color{mauve},
  breaklines=true,
  breakatwhitespace=true,
  commentstyle=\color{textred}\itshape,
  tabsize=3
}



% Content start below
\begin{document}

\author{陈铭涛\\16340024}
\title{数据挖掘周报 4}
% \date{\vspace{-5ex}}
\maketitle

\medskip

% ========== Begin answering questions here

\section{这两周所做工作}
\begin{enumerate}
    \item 学习数据挖掘中的机器学习算法原理,学习了决策树、集成方法的实现和调参方法,了解了文本处理中 RNN、LSTM 神经网络的原理和在 PyTorch 下的具体使用。
    \item 学习了 TalkingData Mobile User Demographics 比赛中的 Kernel,了解其特征工程,构建训练特征集以及模型选择的思路,比如使用 TfIdf 等 NLP 中的方法来处理手机品牌的数据,在华为比赛中也可以用于处理应用类型等类似的字段。
    \item 查看了华为算法精英赛的「用户人口属性预测」比赛的比赛内容和数据
\end{enumerate}

\section{遇到的问题}
\begin{enumerate}
    \item 华为比赛中数据量较大,有一个 25.66G 的表,使用时不能全部加载入内存,目前只在忽略该表的前提下进行比赛。
\end{enumerate}


\section{下一阶段计划}

\begin{enumerate}
    \item 参加华为比赛,继续学习 TalkingData Mobile User Demographics 比赛中不同的 Kernel 做法。
\end{enumerate}

\end{document}
\grid
\grid
