\documentclass[12pt]{article}
% Chinese Support
\usepackage{xeCJK}
\setCJKmainfont{STSong}
% \setmainfont{Times New Roman}

\usepackage{float}

% for symbol
\usepackage{gensymb}
% matrix
\usepackage{amsmath}

% For images
\usepackage{graphicx}
\graphicspath{ {./screenshoot/} }

\newcommand{\numpy}{{\tt numpy}}    % tt font for numpy

\topmargin -1.in
\textheight 9in
\oddsidemargin -.25in
\evensidemargin -.25in
\textwidth 7in


\usepackage{listings}
\usepackage{color}
\usepackage{xcolor}

\definecolor{dkgreen}{rgb}{0,0.6,0}
\definecolor{gray}{rgb}{0.5,0.5,0.5}
\definecolor{mauve}{rgb}{0.58,0,0.82}
\definecolor{textblue}{rgb}{.2,.2,.7}
\definecolor{textred}{rgb}{0.77,0,0}
\definecolor{textgreen}{rgb}{0,0.43,0}

\NeedsTeXFormat{LaTeX2e}[1994/06/01]
\ProvidesPackage{listings-rust}[2018/01/23 Custom Package]

\RequirePackage{color}
\RequirePackage{listings}

\lstdefinelanguage{Rust}{%
  sensitive%
, morecomment=[l]{//}%
, morecomment=[s]{/*}{*/}%
, moredelim=[s][{\itshape\color[rgb]{0,0,0.75}}]{\#[}{]}%
, morestring=[b]{"}%
, alsodigit={}%
, alsoother={}%
, alsoletter={!}%
%
%
% [1] reserve keywords
% [2] traits
% [3] primitive types
% [4] type and value constructors
% [5] identifier
%
, morekeywords={break, continue, else, for, if, in, loop, match, return, while}  % control flow keywords
, morekeywords={as, const, let, move, mut, ref, static}  % in the context of variables
, morekeywords={dyn, enum, fn, impl, Self, self, struct, trait, type, union, use, where}  % in the context of declarations
, morekeywords={crate, extern, mod, pub, super}  % in the context of modularisation
, morekeywords={unsafe}  % markers
, morekeywords={abstract, alignof, become, box, do, final, macro, offsetof, override, priv, proc, pure, sizeof, typeof, unsized, virtual, yield}  % reserved identifiers
%
% grep 'pub trait [A-Za-z][A-Za-z0-9]*' -r . | sed 's/^.*pub trait \([A-Za-z][A-Za-z0-9]*\).*/\1/g' | sort -u | tr '\n' ',' | sed 's/^\(.*\),$/{\1}\n/g' | sed 's/,/, /g'
, morekeywords=[2]{Add, AddAssign, Any, AsciiExt, AsInner, AsInnerMut, AsMut, AsRawFd, AsRawHandle, AsRawSocket, AsRef, Binary, BitAnd, BitAndAssign, Bitor, BitOr, BitOrAssign, BitXor, BitXorAssign, Borrow, BorrowMut, Boxed, BoxPlace, BufRead, BuildHasher, CastInto, CharExt, Clone, CoerceUnsized, CommandExt, Copy, Debug, DecodableFloat, Default, Deref, DerefMut, DirBuilderExt, DirEntryExt, Display, Div, DivAssign, DoubleEndedIterator, DoubleEndedSearcher, Drop, EnvKey, Eq, Error, ExactSizeIterator, ExitStatusExt, Extend, FileExt, FileTypeExt, Float, Fn, FnBox, FnMut, FnOnce, Freeze, From, FromInner, FromIterator, FromRawFd, FromRawHandle, FromRawSocket, FromStr, FullOps, FusedIterator, Generator, Hash, Hasher, Index, IndexMut, InPlace, Int, Into, IntoCow, IntoInner, IntoIterator, IntoRawFd, IntoRawHandle, IntoRawSocket, IsMinusOne, IsZero, Iterator, JoinHandleExt, LargeInt, LowerExp, LowerHex, MetadataExt, Mul, MulAssign, Neg, Not, Octal, OpenOptionsExt, Ord, OsStrExt, OsStringExt, Packet, PartialEq, PartialOrd, Pattern, PermissionsExt, Place, Placer, Pointer, Product, Put, RangeArgument, RawFloat, Read, Rem, RemAssign, Seek, Shl, ShlAssign, Shr, ShrAssign, Sized, SliceConcatExt, SliceExt, SliceIndex, Stats, Step, StrExt, Sub, SubAssign, Sum, Sync, TDynBenchFn, Terminal, Termination, ToOwned, ToSocketAddrs, ToString, Try, TryFrom, TryInto, UnicodeStr, Unsize, UpperExp, UpperHex, WideInt, Write}
, morekeywords=[2]{Send}  % additional traits
%
, morekeywords=[3]{bool, char, f32, f64, i8, i16, i32, i64, isize, str, u8, u16, u32, u64, unit, usize, i128, u128}  % primitive types
%
, morekeywords=[4]{Err, false, None, Ok, Some, true}  % prelude value constructors
% grep 'pub \(type\|struct\|enum\) [A-Za-z][A-Za-z0-9]*' -r . | sed 's/^.*pub \(type\|struct\|enum\) \([A-Za-z][A-Za-z0-9]*\).*/\2/g' | sort -u | tr '\n' ',' | sed 's/^\(.*\),$/{\1}\n/g' | sed 's/,/, /g'    
, morekeywords=[3]{AccessError, Adddf3, AddI128, AddoI128, AddoU128, ADDRESS, ADDRESS64, addrinfo, ADDRINFOA, AddrParseError, Addsf3, AddU128, advice, aiocb, Alignment, AllocErr, AnonPipe, Answer, Arc, Args, ArgsInnerDebug, ArgsOs, Argument, Arguments, ArgumentV1, Ashldi3, Ashlti3, Ashrdi3, Ashrti3, AssertParamIsClone, AssertParamIsCopy, AssertParamIsEq, AssertUnwindSafe, AtomicBool, AtomicPtr, Attr, auxtype, auxv, BackPlace, BacktraceContext, Barrier, BarrierWaitResult, Bencher, BenchMode, BenchSamples, BinaryHeap, BinaryHeapPlace, blkcnt, blkcnt64, blksize, BOOL, boolean, BOOLEAN, BoolTrie, BorrowError, BorrowMutError, Bound, Box, bpf, BTreeMap, BTreeSet, Bucket, BucketState, Buf, BufReader, BufWriter, Builder, BuildHasherDefault, BY, BYTE, Bytes, CannotReallocInPlace, cc, Cell, Chain, CHAR, CharIndices, CharPredicateSearcher, Chars, CharSearcher, CharsError, CharSliceSearcher, CharTryFromError, Child, ChildPipes, ChildStderr, ChildStdin, ChildStdio, ChildStdout, Chunks, ChunksMut, ciovec, clock, clockid, Cloned, cmsgcred, cmsghdr, CodePoint, Color, ColorConfig, Command, CommandEnv, Component, Components, CONDITION, condvar, Condvar, CONSOLE, CONTEXT, Count, Cow, cpu, CRITICAL, CStr, CString, CStringArray, Cursor, Cycle, CycleIter, daddr, DebugList, DebugMap, DebugSet, DebugStruct, DebugTuple, Decimal, Decoded, DecodeUtf16, DecodeUtf16Error, DecodeUtf8, DefaultEnvKey, DefaultHasher, dev, device, Difference, Digit32, DIR, DirBuilder, dircookie, dirent, dirent64, DirEntry, Discriminant, DISPATCHER, Display, Divdf3, Divdi3, Divmoddi4, Divmodsi4, Divsf3, Divsi3, Divti3, dl, Dl, Dlmalloc, Dns, DnsAnswer, DnsQuery, dqblk, Drain, DrainFilter, Dtor, Duration, DwarfReader, DWORD, DWORDLONG, DynamicLibrary, Edge, EHAction, EHContext, Elf32, Elf64, Empty, EmptyBucket, EncodeUtf16, EncodeWide, Entry, EntryPlace, Enumerate, Env, epoll, errno, Error, ErrorKind, EscapeDebug, EscapeDefault, EscapeUnicode, event, Event, eventrwflags, eventtype, ExactChunks, ExactChunksMut, EXCEPTION, Excess, ExchangeHeapSingleton, exit, exitcode, ExitStatus, Failure, fd, fdflags, fdsflags, fdstat, ff, fflags, File, FILE, FileAttr, filedelta, FileDesc, FilePermissions, filesize, filestat, FILETIME, filetype, FileType, Filter, FilterMap, Fixdfdi, Fixdfsi, Fixdfti, Fixsfdi, Fixsfsi, Fixsfti, Fixunsdfdi, Fixunsdfsi, Fixunsdfti, Fixunssfdi, Fixunssfsi, Fixunssfti, Flag, FlatMap, Floatdidf, FLOATING, Floatsidf, Floatsisf, Floattidf, Floattisf, Floatundidf, Floatunsidf, Floatunsisf, Floatuntidf, Floatuntisf, flock, ForceResult, FormatSpec, Formatted, Formatter, Fp, FpCategory, fpos, fpos64, fpreg, fpregset, FPUControlWord, Frame, FromBytesWithNulError, FromUtf16Error, FromUtf8Error, FrontPlace, fsblkcnt, fsfilcnt, fsflags, fsid, fstore, fsword, FullBucket, FullBucketMut, FullDecoded, Fuse, GapThenFull, GeneratorState, gid, glob, glob64, GlobalDlmalloc, greg, group, GROUP, Guard, GUID, Handle, HANDLE, Handler, HashMap, HashSet, Heap, HINSTANCE, HMODULE, hostent, HRESULT, id, idtype, if, ifaddrs, IMAGEHLP, Immut, in, in6, Incoming, Infallible, Initializer, ino, ino64, inode, input, InsertResult, Inspect, Instant, int16, int32, int64, int8, integer, IntermediateBox, Internal, Intersection, intmax, IntoInnerError, IntoIter, IntoStringError, intptr, InvalidSequence, iovec, ip, IpAddr, ipc, Ipv4Addr, ipv6, Ipv6Addr, Ipv6MulticastScope, Iter, IterMut, itimerspec, itimerval, jail, JoinHandle, JoinPathsError, KDHELP64, kevent, kevent64, key, Key, Keys, KV, l4, LARGE, lastlog, launchpad, Layout, Lazy, lconv, Leaf, LeafOrInternal, Lines, LinesAny, LineWriter, linger, linkcount, LinkedList, load, locale, LocalKey, LocalKeyState, Location, lock, LockResult, loff, LONG, lookup, lookupflags, LookupHost, LPBOOL, LPBY, LPBYTE, LPCSTR, LPCVOID, LPCWSTR, LPDWORD, LPFILETIME, LPHANDLE, LPOVERLAPPED, LPPROCESS, LPPROGRESS, LPSECURITY, LPSTARTUPINFO, LPSTR, LPVOID, LPWCH, LPWIN32, LPWSADATA, LPWSAPROTOCOL, LPWSTR, Lshrdi3, Lshrti3, lwpid, M128A, mach, major, Map, mcontext, Metadata, Metric, MetricMap, mflags, minor, mmsghdr, Moddi3, mode, Modsi3, Modti3, MonitorMsg, MOUNT, mprot, mq, mqd, msflags, msghdr, msginfo, msglen, msgqnum, msqid, Muldf3, Mulodi4, Mulosi4, Muloti4, Mulsf3, Multi3, Mut, Mutex, MutexGuard, MyCollection, n16, NamePadding, NativeLibBoilerplate, nfds, nl, nlink, NodeRef, NoneError, NonNull, NonZero, nthreads, NulError, OccupiedEntry, off, off64, oflags, Once, OnceState, OpenOptions, Option, Options, OptRes, Ordering, OsStr, OsString, Output, OVERLAPPED, Owned, Packet, PanicInfo, Param, ParseBoolError, ParseCharError, ParseError, ParseFloatError, ParseIntError, ParseResult, Part, passwd, Path, PathBuf, PCONDITION, PCONSOLE, Peekable, PeekMut, Permissions, PhantomData, pid, Pipes, PlaceBack, PlaceFront, PLARGE, PoisonError, pollfd, PopResult, port, Position, Powidf2, Powisf2, Prefix, PrefixComponent, PrintFormat, proc, Process, PROCESS, processentry, protoent, PSRWLOCK, pthread, ptr, ptrdiff, PVECTORED, Queue, radvisory, RandomState, Range, RangeFrom, RangeFull, RangeInclusive, RangeMut, RangeTo, RangeToInclusive, RawBucket, RawFd, RawHandle, RawPthread, RawSocket, RawTable, RawVec, Rc, ReadDir, Receiver, recv, RecvError, RecvTimeoutError, ReentrantMutex, ReentrantMutexGuard, Ref, RefCell, RefMut, REPARSE, Repeat, Result, Rev, Reverse, riflags, rights, rlim, rlim64, rlimit, rlimit64, roflags, Root, RSplit, RSplitMut, RSplitN, RSplitNMut, RUNTIME, rusage, RwLock, RWLock, RwLockReadGuard, RwLockWriteGuard, sa, SafeHash, Scan, sched, scope, sdflags, SearchResult, SearchStep, SECURITY, SeekFrom, segment, Select, SelectionResult, sem, sembuf, send, Sender, SendError, servent, sf, Shared, shmatt, shmid, ShortReader, ShouldPanic, Shutdown, siflags, sigaction, SigAction, sigevent, sighandler, siginfo, Sign, signal, signalfd, SignalToken, sigset, sigval, Sink, SipHasher, SipHasher13, SipHasher24, size, SIZE, Skip, SkipWhile, Slice, SmallBoolTrie, sockaddr, SOCKADDR, sockcred, Socket, SOCKET, SocketAddr, SocketAddrV4, SocketAddrV6, socklen, speed, Splice, Split, SplitMut, SplitN, SplitNMut, SplitPaths, SplitWhitespace, spwd, SRWLOCK, ssize, stack, STACKFRAME64, StartResult, STARTUPINFO, stat, Stat, stat64, statfs, statfs64, StaticKey, statvfs, StatVfs, statvfs64, Stderr, StderrLock, StderrTerminal, Stdin, StdinLock, Stdio, StdioPipes, Stdout, StdoutLock, StdoutTerminal, StepBy, String, StripPrefixError, StrSearcher, subclockflags, Subdf3, SubI128, SuboI128, SuboU128, subrwflags, subscription, Subsf3, SubU128, Summary, suseconds, SYMBOL, SYMBOLIC, SymmetricDifference, SyncSender, sysinfo, System, SystemTime, SystemTimeError, Take, TakeWhile, tcb, tcflag, TcpListener, TcpStream, TempDir, TermInfo, TerminfoTerminal, termios, termios2, TestDesc, TestDescAndFn, TestEvent, TestFn, TestName, TestOpts, TestResult, Thread, threadattr, threadentry, ThreadId, tid, time, time64, timespec, TimeSpec, timestamp, timeval, timeval32, timezone, tm, tms, ToLowercase, ToUppercase, TraitObject, TryFromIntError, TryFromSliceError, TryIter, TryLockError, TryLockResult, TryRecvError, TrySendError, TypeId, U64x2, ucontext, ucred, Udivdi3, Udivmoddi4, Udivmodsi4, Udivmodti4, Udivsi3, Udivti3, UdpSocket, uid, UINT, uint16, uint32, uint64, uint8, uintmax, uintptr, ulflags, ULONG, ULONGLONG, Umoddi3, Umodsi3, Umodti3, UnicodeVersion, Union, Unique, UnixDatagram, UnixListener, UnixStream, Unpacked, UnsafeCell, UNWIND, UpgradeResult, useconds, user, userdata, USHORT, Utf16Encoder, Utf8Error, Utf8Lossy, Utf8LossyChunk, Utf8LossyChunksIter, utimbuf, utmp, utmpx, utsname, uuid, VacantEntry, Values, ValuesMut, VarError, Variables, Vars, VarsOs, Vec, VecDeque, vm, Void, WaitTimeoutResult, WaitToken, wchar, WCHAR, Weak, whence, WIN32, WinConsole, Windows, WindowsEnvKey, winsize, WORD, Wrapping, wrlen, WSADATA, WSAPROTOCOL, WSAPROTOCOLCHAIN, Wtf8, Wtf8Buf, Wtf8CodePoints, xsw, xucred, Zip, zx}
%
, morekeywords=[5]{assert!, assert_eq!, assert_ne!, cfg!, column!, compile_error!, concat!, concat_idents!, debug_assert!, debug_assert_eq!, debug_assert_ne!, env!, eprint!, eprintln!, file!, format!, format_args!, include!, include_bytes!, include_str!, line!, module_path!, option_env!, panic!, print!, println!, select!, stringify!, thread_local!, try!, unimplemented!, unreachable!, vec!, write!, writeln!}  % prelude macros
}%

\lstdefinestyle{colouredRust}%
{ basicstyle=\ttfamily%
, identifierstyle=%
, commentstyle=\color[gray]{0.4}%
, stringstyle=\color[rgb]{0, 0, 0.5}%
, keywordstyle=\bfseries% reserved keywords
, keywordstyle=[2]\color[rgb]{0.75, 0, 0}% traits
, keywordstyle=[3]\color[rgb]{0, 0.5, 0}% primitive types
, keywordstyle=[4]\color[rgb]{0, 0.5, 0}% type and value constructors
, keywordstyle=[5]\color[rgb]{0, 0, 0.75}% macros
, columns=spaceflexible%
, keepspaces=true%
, showspaces=false%
, showtabs=false%
, showstringspaces=true%
}%

\lstdefinestyle{boxed}{
  style=colouredRust%
, numbers=left%
, firstnumber=auto%
, numberblanklines=true%
, frame=trbL%
, numberstyle=\tiny%
, frame=leftline%
, numbersep=7pt%
, framesep=5pt%
, framerule=10pt%
, xleftmargin=15pt%
, backgroundcolor=\color[gray]{0.97}%
, rulecolor=\color[gray]{0.90}%
}

% \setmonofont{FiraCode-Regular}
\lstset{frame=tb,
  language=Rust,
  aboveskip=3mm,
  belowskip=3mm,
  showstringspaces=false,
  columns=flexible,
  basicstyle={\ttfamily},
  numbers=none,
  numberstyle=\tiny\color{gray},
  keywordstyle=\color{blue}\itshape,
  stringstyle=\color{mauve},
  breaklines=true,
  breakatwhitespace=true,
  commentstyle=\color{textred}\itshape,
  tabsize=3
}
\usepackage{indentfirst}
\usepackage{bm}
\usepackage[ruled]{algorithm2e}


% Content start below
\begin{document}

\author{陈铭涛\\16340024}
\title{数据挖掘第二次项目\\
    \large{并行决策树集成}}
% \date{\vspace{-5ex}}
\maketitle

\medskip

% ========== Begin answering questions here

\section{CART 算法}

决策树是一种树结构,每一个非叶子节点表示一个对一个特征的分裂,叶子节点存放了分类问题中的类别和回归问题中的数值。
其对样本进行预测的方法是从根节点开始,根据每一个分支节点的特征属性,决定输出方向直到到达叶子节点,将叶子节点的数值作为输出值。

决策树的优点为决策过程较为容易令人理解,可解释性强。决策树的构造方法包括了 ID3, C4.5等算法,在本次项目中主要实现的是 CART 算法。

CART 包含了分类决策树和回归决策树的算法,本次实现了其中的回归树算法。CART 构造出的决策树为一棵二叉树。对于分类问题,CART 算法通过 Gini Index 来计算数据集的纯度,以决定一个节点的分裂,其公式如下:

假设样本集合$D$中第$k$类样本比例为$p_k$($k=1,2...|N|$),
\begin{equation}
    \begin{aligned}
    Gini(D)=1-\sum_{k=1}^{|N|}p_k^2
    \end{aligned}
\end{equation}

$Gini(D)$反映了从$D$中随机抽取两个样本其类别不同的概率,越小则代表样本纯度越高。对于属性集合$A$上的属性$a$,将所有$D$上取值为$a^v$的样本记为$D^v$,则$a$上的Gini Index为:
\begin{equation}
    Gini(D, a)=\sum_{v=1}^{|V|}\frac{|D^v|}{|D|}Gini(D^v)
\end{equation}
选择$A$中的划分点即为
\begin{equation}
    a_*=\mathop{\arg\min}_{a\in A}Gini(D,a)
\end{equation}

当要解决的问题是回归问题时,假设样本集合$D$中第$i$个样本的标签为$y_i$,则最小化的目标为回归标签的平方误差和,即
\begin{equation}
    SSE(D)=\sum_{i=1}^{|N|}(y_i-\overline{y})^2
\end{equation}

对于在属性$a$上的$v$值分裂的节点,设其左子树和右子树上的样本集合分别为$D^L$, $D^R$, 则其平方误差和为:
\begin{equation}
    SSE(D, a, v)=SSE(D^L) + SSE(D^R)
\end{equation}

则对于属性$a$上的划分点的选择为:
\begin{equation}
    a_*=\mathop{\arg\min}_{a\in A, v \in V}SSE(D,a,v)
\end{equation}

CART 算法进行决策树构建的方法如下:

\begin{algorithm}[H]
    \SetAlgoLined
    \KwResult{分类决策树或回归决策树 }
    从深度为0开始构建\;
    对各特征列进行排序\;

    \While{未达到终止条件}{
      instructions\;
      \eIf{condition}{
       instructions1\;
       instructions2\;
       }{
       instructions3\;
      }
     }
     \caption{CART 决策树构建}
\end{algorithm}

由于对于连续变量的特征,$|V|$可能是一个较大的数值

\section{Gradient Boosting}

使用的损失函数为均方误差:
\begin{equation}
    \begin{aligned}
    L= \frac{1}{N}\sum_{u=1}^{n}(Y_i - \hat{Y_i})\\
    \end{aligned}
\end{equation}

\section{Random Forest}

\section{代码实现}

出于内存、速度和并行化实现的考虑,本次项目选择了使用 Rust 语言实现,原因是 Rust 的 RAII 机制使得资源可以及时地释放,提升内存利用率;由编译器提供的静态检查可以避免多线程时线程不安全的情况,降低 debug 难度;作为通过 LLVM 后端编译为机器代码的静态语言 Rust 可以在相同的实现下获得比 Python 更高的速度。使用的编译器版本为\lstinline{rustc 1.35.0}.

程序实现中使用的第三方库如下:
\begin{enumerate}
    \item rayon: 提供基于迭代器的便捷地编写并行代码的方法
    \item rand: 提供随机数生成
    \item csv: 提供对 csv 文件的读取
    \item indicatif: 提供命令行进度条实现
    \item ndarray: 提供类似 numpy 的多维数组的操作
    \item num-traits: 提供数值类型上的一些实用方法,如最大最小值等
    \item log: 程序日志
    \item pretty\_env\_logger: 程序日志输出
    \item num\_cpus: 获取系统 CPU 核心数量
    \item serde: 提供将结构体变量序列化的功能
    \item serde\_json: 用于以 json 格式将序列化后的模型变量保存至文件
\end{enumerate}

项目中包括的主要代码文件如下
\begin{enumerate}
    \item[$\bullet$] data\_frame.rs: 使用一个二维的 \lstinline{ndarray} 作为程序使用的 \lstinline{DataFrame} 类型,并定义了类型别名\lstinline{V} 作为全局的数据存储类型,可设为 \lstinline{f64} 或 \lstinline{f32}, 此外还包含了 csv 文件读写等其他实用函数
    \item[$\bullet$] learner.rs: 定义了一个 \lstinline{Learner} trait,包括了类似于 sklearn 的\lstinline{fit} 和\lstinline{predict} 两个方法,决策树,Boosting和 Random Forest 都需实现该 trait。
    \item[$\bullet$] tree.rs: 包括了 \lstinline{DecisionTree}类型,实现了 CART 算法,可对单个决策树进行训练和预测。
    \item[$\bullet$] boosting.rs: 包括了\lstinline{GradientBoosting} 类型的定义与训练和预测的实现。
    \item[$\bullet$] random\_forest.rs: 包括了\lstinline{RandomForest} 类型的定义与训练和预测的实现。
    \item[$\bullet$] utils 目录:包括了数个实用功能,如获取数据列排序序列,交叉验证,模型分数计算等。
\end{enumerate}

bin 目录下的包含 main 函数的代码文件如下:
\begin{enumerate}
    \item[$\bullet$] boost\_cv.rs:使用 GDBT 进行交叉验证
    \item[$\bullet$] boost\_predict.rs:使用 GDBT 进行训练并输出在测试集上的预测结果
    \item[$\bullet$] cv.rs:使用单棵决策树进行交叉验证
    \item[$\bullet$] parallel\_performance.rs:接收一个命令行整数作为程序最多使用的线程数,进行一次单棵决策树的训练,用于测试并行化的效率
    \item[$\bullet$] predict.rs:使用单棵决策树进行训练并输出测试集预测结果
    \item[$\bullet$] rf\_cv.rs:使用随机森林进行交叉验证
    \item[$\bullet$] rf\_predict.rs:使用随机森林进行训练并输出测试集预测结果
\end{enumerate}

运行任一 bin 目录下的代码的方法为在项目目录下执行命令(*nix系统下),其中\lstinline{EMTM_LOG=info}的作用是使程序将日志输出到命令行:
\begin{lstlisting}
EMTM_LOG=info cargo run --release --bin {EXEC_NAME}
\end{lstlisting}

所有代码都需在 release 模式下进行编译,否则速度可能会有20到100倍的减慢。


\section{并行化表现}

由于并行化的实现主要位于单棵决策树的训练中,对于并行化表现的测试主要针对单棵决策树的训练与预测。

在命令行下使用如下命令测试了从单线程到12线程下对单棵决策树训练时的时间:
\begin{lstlisting}
for ((i = 1; i<=12;i++)) cargo run --release --bin parallel_performance $i >>../threads_performance.txt
\end{lstlisting}

将获得的训练速度相对单线程下训练速度的提升数据进行绘图如下:
\begin{figure}[H]
    \centering
    \includegraphics[scale=0.35]{train_time_speedup.png}
    \caption{并行训练速度提升情况}
    \label{}
\end{figure}

预测速度的数据进行绘图如下:
\begin{figure}[H]
    \centering
    \includegraphics[scale=0.35]{test_time_speedup.png}
    \caption{并行预测速度提升情况}
    \label{}
\end{figure}

可见当线程数提升到一定程度下对训练速度的提升大约收敛在 4 倍,而对预测速度的提升不大。

\section{验证}
以下程序测试均在一台搭载6核12线程 CPU,运行 macOS 系统的笔记本电脑上运行。
\\

验证的标准为$R^2$, 其计算方法如下:

\begin{equation}
    \begin{aligned}
    \bar{y}= \frac{1}{n}\sum_{i=1}^n y_i,\\
    SS_{tot} = \sum_i(y_i-\overline{y})^2,\\
    SS_{res} = \sum_i(y_i-f_i)^2,\\
    R^2 = 1 - \frac{SS_{res}}{SS_{tot}}
    \end{aligned}
\end{equation}

其中$y_i$为第$i$个样本的实际观察值,$f_i$为第$i$个样本的模型预测值。$R^2$的取值通常在0与1之间,越接近1代表预测值与真实值匹配程度越高。


使用 \lstinline{LightGBM} 默认参数构建一个模型运行3折交叉验证进行对比:

\begin{figure}[H]
    \centering
    \includegraphics[scale=0.6]{lgb-baseline.png}
    \caption{LightGBM 交叉验证结果}
    \label{}
\end{figure}

对单棵决策树进行交叉验证获得的结果如下:

在训练集上获得的平均$R^2$分数为0.14947528261278345

在验证集上获得的平均$R^2$分数为0.143995380629807

训练时间平均为 682423 ms.

\begin{figure}[H]
    \centering
    \includegraphics[scale=0.6]{single-tree-cv.png}
    \caption{单棵决策树训练交叉验证结果}
    \label{}
\end{figure}

使用 Gradient Boosting 训练150步,设置单棵树最大生长至 2 层,进行交叉验证获得的结果如下:

在训练集上获得的平均$R^2$分数为0.14595633826167811

在验证集上获得的平均$R^2$分数为0.14329480642795986

训练时间平均为 223031 ms.

\begin{figure}[H]
    \centering
    \includegraphics[scale=0.6]{gbdt-cv.png}
    \caption{GBDT 训练交叉验证结果}
    \label{}
\end{figure}

使用 Random Forest 训练,决策树数量为150,不限制决策树生长深度,进行交叉验证获得的结果如下:

在训练集上获得的平均$R^2$分数为0.1496125066152948

在验证集上获得的平均$R^2$分数为0.14889900127096856

训练时间平均为 444804 ms.

\begin{figure}[H]
    \centering
    \includegraphics[scale=0.6]{rf-cv.png}
    \caption{随机森林训练交叉验证结果}
    \label{}
\end{figure}

四个训练中使用取样工具查看内存占用值分别如下:
\begin{enumerate}
    \item LightGBM: 5.2G
    \begin{figure}[H]
        \centering
        \includegraphics[scale=0.46]{lgb-memory.png}
        \caption{LightGBM 内存占用}
        \label{}
    \end{figure}

    \item 单决策树:3.2G
    \begin{figure}[H]
        \centering
        \includegraphics[scale=0.5]{single-tree-memory.png}
        \caption{单棵决策树训练内存占用}
        \label{}
    \end{figure}

    \item Gradient Boosting: 3.4G
    \begin{figure}[H]
        \centering
        \includegraphics[scale=0.44]{gbdt-memory.png}
        \caption{Gradient Boosting训练内存占用}
        \label{}
    \end{figure}

    \item Random Forest: 1.6G
    \begin{figure}[H]
        \centering
        \includegraphics[scale=0.44]{rf-memory.png}
        \caption{Random Forest训练内存占用}
        \label{}
    \end{figure}
\end{enumerate}

\section{Kaggle 分数}

使用单棵决策树训练至10层后提交至 Kaggle 获得的分数为0.16087:
\begin{figure}[H]
    \centering
    \includegraphics[scale=0.5]{single-tree-300-10-kaggle.png}
    \caption{单棵决策树分数}
    \label{}
\end{figure}

使用 Gradient Boosting, Learning Rate 固定为0.25,基学习器最大训练至3层,训练步数为400时的分数为0.16957:
\begin{figure}[H]
    \centering
    \includegraphics[scale=0.5]{gbdt-fixed-lr-kaggle.png}
    \caption{lr=0.25, GBDT}
    \label{}
\end{figure}

使用 Random Forest, 不限制单棵决策树生长,使用决策树总数为350,时的分数为0.17210:
\begin{figure}[H]
    \centering
    \includegraphics[scale=0.5]{RF-400-4-005-350.png}
    \caption{随机森林}
    \label{}
\end{figure}

\end{document}
\grid
\grid
