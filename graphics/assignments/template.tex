\documentclass[12pt]{article}
% Chinese Support
\usepackage{xeCJK}
\setCJKmainfont{STSong}
% \setmainfont{Times New Roman}


% for symbol
\usepackage{gensymb}
% matrix
\usepackage{amsmath}


% For images
\usepackage{graphicx}
\graphicspath{ {./screenshoot/} }

\newcommand{\numpy}{{\tt numpy}}    % tt font for numpy

\topmargin -1.in
\textheight 9in
\oddsidemargin -.25in
\evensidemargin -.25in
\textwidth 7in


\usepackage{listings}
\usepackage{color}

\definecolor{dkgreen}{rgb}{0,0.6,0}
\definecolor{gray}{rgb}{0.5,0.5,0.5}
\definecolor{mauve}{rgb}{0.58,0,0.82}

\setmonofont{FiraCode-Regular}
\lstset{frame=tb,
  language=C++,
  aboveskip=3mm,
  belowskip=3mm,
  showstringspaces=false,
  columns=flexible,
  basicstyle={\small\ttfamily},
  numbers=none,
  numberstyle=\tiny\color{gray},
  keywordstyle=\color{blue},
  commentstyle=\color{dkgreen},
  stringstyle=\color{mauve},
  breaklines=true,
  breakatwhitespace=true,
  tabsize=3
}



% Content start below
\begin{document}

\author{陈铭涛}
\title{Title}
% \date{\vspace{-5ex}}
\maketitle

\medskip

% ========== Begin answering questions here

\section{Answer to question 1:}

% ========== Just examples, please delete before submitting
Use inline equations for simple math $1+1=2$, and centered equations for more involved or important equations
\begin{equation}
    a^2 + b^2 = c^2.
\end{equation}

Some people like to write scalars without boldface $x+y=1$ and vectors or matrices in boldface
\begin{equation}
    \mathbf{A} \mathbf{x} = \mathbf{b}.
\end{equation}

An example of a matrix \LaTeX:
\begin{equation}
    \mathbf{A} = \left(
    \begin{array}{ccc}
    3 & -1 & 2 \\ 	
    0 & 1 & 2 \\ 
    1 & 0 & -1 \\
\end{array} 
\right).  
\end{equation}

With a labeled equation such as the following:
\begin{equation}
    \label{accel}
    \frac{d^2 x}{d t^2} = a
\end{equation}
you can referrer to the equation later. In equation \ref{accel} we defined acceleration.

\section{Answer to question 2}

% Code below
\begin{lstlisting}

\end{lstlisting}
\begin{center}
    \includegraphics[scale=2]{logo.png}
\end{center}


\end{document}
\grid
\grid
